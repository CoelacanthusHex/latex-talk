\section{字体排印}

\begin{frame}{先看一个例子}
\begin{minipage}{\textwidth}
  \fontspec{SimHei}[AutoFakeBold, AutoFakeSlant]\small
  \hyphenpenalty=10000\hbadness=10000\linespread{0.85}\selectfont
  \textbf{Typography} is the art and technique of arranging type to make written language
  legible,readable,and appealing when displayed. The arrangement of type involves selecting
  typefaces, point sizes, line lengths, line-spacing(\textit{leading}), and letter-spacing
  (\textit{tracking}), and adjusting the space between pairs of letters(\textit{kerning}).
  The term typography is also applied to the style,arrangement, and appearance of the letters,
  numbers, and symbols created by the process. \textbf{Type design} is a closely related craft,
  sometimes considered part of typography;most typographers do not design typefaces, and some
  type designers do not consider themselves typographers. Typography also may be used as a
  decorative device,unrelated to communication of information.
\end{minipage}
\let\thefootnote\relax
\footnotetext{中易黑体,行距 0.85 倍,关闭 hypenation}
\end{frame}

\begin{frame}{没有对比就没有伤害}
\begin{minipage}{\textwidth}
  \fontspec{EB Garamond}\small
  \textbf{Typography} is the art and technique of arranging type to make written language
  legible, readable, and appealing when displayed. The arrangement of type involves selecting
  typefaces, point sizes, line lengths, line-spacing (\textit{leading}), and letter-spacing
  (\textit{tracking}), and adjusting the space between pairs of letters (\textit{kerning}).
  The term typography is also applied to the style, arrangement, and appearance of the letters,
  numbers, and symbols created by the process. \textbf{Type design} is a closely related craft,
  sometimes considered part of typography; most typographers do not design typefaces, and some
  type designers do not consider themselves typographers. Typography also may be used as a
  decorative device, unrelated to communication of information.
\end{minipage}
\let\thefootnote\relax
\footnotetext{EB Garamond,默认设置}
\end{frame}

\begin{frame}{术语}
\footnotesize
\begin{columns}[t]
\begin{column}{0.48\textwidth}
  \begin{itemize}
    \item 语言\zhparen{language}
    \item 文字\zhparen{script}
    \item 书写系统\zhparen{writting system}
  \end{itemize}
  \begin{itemize}
    \item 符号\zhparen{symbol}
    \item 字符\zhparen{character}
    \item 字符形\zhparen{glyph}
  \end{itemize}
  \begin{itemize}
    \item 字符集\zhparen{character set}
    \item 编码\zhparen{encoding}
    \item 码位\zhparen{code point}
  \end{itemize}
\end{column}
\begin{column}{0.48\textwidth}
  \begin{itemize}
    \item 字体\zhparen{font}
    \item 字型\zhparen{typeface}
  \end{itemize}
  \begin{itemize}
    \item 易认性\zhparen{legibility}
    \item 可读性\zhparen{readability}
  \end{itemize}
  \begin{itemize}
    \item 字偶间距\zhparen{kerning}
    \item 字距\zhparen{tracking}
  \end{itemize}
  \begin{itemize}
    \item 栅格化\zhparen{rasterization}
    \item 渲染提示\zhparen{hinting}
  \end{itemize}
\end{column}
\end{columns}
\end{frame}

\begin{frame}[standout]
  \large \textbf{\LaTeX{} will do (almost) all the things for you.}
\end{frame}

\begin{frame}[fragile]
\frametitle{Punctuations: hyphen/dash}
\begin{columns}
\begin{column}{0.84\textwidth}
  \begin{itemize}
    \item Hyphen \enparen{\usv{002D}}: |-|
      \begin{itemize}
        \item Four-dimensional momentum
        \item Hyphenation
      \end{itemize}
    \item En dash \enparen{\usv{2013}}: |--|
      \begin{itemize}
        \item Newton--Leibniz formula (\emph{cf.} Levi-Civita symbol)
        \item pp.~187--189
      \end{itemize}
    \item Em dash \enparen{\usv{2014}}: |---|
      \begin{itemize}
        \item Red, white, and blue---these are the colors of the flag
        \item Like colon, parentheses, \emph{etc.}
      \end{itemize}
    \item Minus \enparen{\usv{2212}}: |$-$|
      \begin{itemize}
        \item $a-b$, $-a$
      \end{itemize}
  \end{itemize}
\end{column}
\begin{column}{0.13\textwidth}
  \tiny\RaggedRight
  A hyphen\alert{-}ation algo\alert{-}rithm is a set of rules, especially one codified
  for imple\alert{-}mentation in a computer program, that decides at which points
  a word can be broken over two lines with a hyphen.
\end{column}
\end{columns}
\end{frame}

\begin{frame}[fragile]
\frametitle{Punctuations: quotation mark}
\begin{itemize}
  \item Left/right, single/double:
    \begin{itemize}
      \item `\ldots{}' \enparen{\usv{2018}, \usv{2019}}: |`...'|
      \item ``\ldots{}'' \enparen{\usv{201C}, \usv{201D}}: |``...''|
    \end{itemize}
  \item Different languages:
    \begin{itemize}
      \item `British ``English'' style' and ``American `English' style''
      \item „German'', ''Finnish'', «French», »Danish«, \emph{etc.}
      \item Use \pkg{csquotes} package
    \end{itemize}
  \item Programming:
    \begin{itemize}
      \item |char* my_name = "Xiangdong Zeng";|
    \end{itemize}
  \item Mathematics:
    \begin{itemize}
      \item |x'| = |x^{\prime}|: $x'(t) = x^{\prime}(t)$
    \end{itemize}
\end{itemize}
\end{frame}

\begin{frame}[fragile]
\frametitle{中文标点符号}
\begin{itemize}
  \item 句号
    \begin{itemize}
      \item 正常文本。科技文本.
    \end{itemize}
  \item 引号
    \begin{itemize}
      \item 『传统风格』,「某乎风格」,“标准风格”,‘奇葩风格’
    \end{itemize}
  \item 破折号
    \begin{itemize}
      \item 断开{\CJKfontspec{Source Han Serif SC Medium}\symbol{"2014}\symbol{"2014}}是不好的,
            不断开——是好的
    \end{itemize}
  \item 波浪号:
    \begin{itemize}
      \item |~| ≠ |\textasciitilde| ≠ |\texttildelow| ≠ |$\sim$| ≠ 你要的那一个
      \item 那就是~青\textasciitilde 藏\texttildelow 高 $\sim$ 原~~~~
        \begin{itemize}
          \item \texttt{\usv{007E}: Tilde}
          \item \texttt{\usv{02F7}: Modifier letter low tilde}
          \item \texttt{\usv{223C}: Tilde operator}
          \item \texttt{\usv{FF5E}: Fullwidth tilde}
          \item \ldots{}
        \end{itemize}
    \end{itemize}
\end{itemize}
\end{frame}

\begin{frame}{使用字体}
\begin{itemize}
  \item 字体家族
    \begin{itemize}
      % TODO
      \item 衬线体:Times New Roman, Garamond, Palatino, \emph{etc.}
      \item 无衬线体:{Helvetica}, {Frutiger}, {Avenir Next}, \emph{etc.}
      \item 等宽体:\texttt{Courier}, \texttt{Consolas}, \texttt{Iosevka}, \emph{etc.}
      \item 中文字体:宋、{\CJKfontspec{Source Han Sans SC}黑}、
        {\CJKfontspec{FZFangSong-Z02}仿}、{\CJKfontspec{FZKai-Z03}楷}……
    \end{itemize}
  \item 样式
    \begin{itemize}
      \item 粗体:\textbf{Bold}
      \item 伪粗体:{\addfontfeatures{AutoFakeBold=3}\textbf{Faked bold}}
      \item 意大利体:\textit{Italic}
      \item 伪斜体:{\addfontfeatures{AutoFakeSlant=0.2}\textsl{Slant}}
    \end{itemize}
  \item OpenType
    \begin{itemize}
      \item 连字\zhparen{ligature}:{f}{f} $\to$ ff, {f}{i} $\to$ fi, {f}{l} $\to$ fl
      \item 老式数字\zhparen{old-style number}:
        0123456789 $\to$ {\addfontfeatures{Numbers=OldStyle}0123456789}
      \item 字偶间距\zhparen{kerning}:{T}{y} $\to$ Ty, {W}{A} $\to$ WA
    \end{itemize}
  \item \alert{请勿同时使用三种以上字体}
\end{itemize}
\end{frame}
