\section{进阶扩展}

\begin{frame}[fragile]
\frametitle{\TeX{} 宏编程}
\begin{itemize}
  \item<+-> \TeX{} 层面

    \begin{itemize}
      \item 守序善良——定义命令:|\def|、|\gdef|、|\let|
      \item 绝对中立——展开控制:|\edef|、|\expandafter|、|\aftergroup|
      \item 混乱邪恶——类别码:|\catcode|
    \end{itemize}

  \item<+-> \LaTeX{} 层面

    \begin{itemize}
      \item 定义新命令:|\newcommand|、|\renewcommand|、|\newenvironment|
      \item 内部命令:|\makeatletter|、|\makeatother|
    \end{itemize}

  \item<+-> \LaTeX3——可望还不可即的未来

    \begin{itemize}
      \item 命令举例:|\cs_new:cpx|、|\seq_sort:Nn|、|\regex_match:nnTF|
      \item 创建用户层命令:\pkg{xparse} 宏包(已默认载入)
      \item 面向对象编程:\pkg{xtemplate} 宏包
      \item \pkg{fontspec}、\pkg{siunitx}、\pkg{ctex}、\pkg{fduthesis} 等均使用 \LaTeX3 实现
      \item \LaTeX{} 内核正逐步迁移至 \LaTeX3
    \end{itemize}

  \item<+-> 外部语言调用

    \begin{itemize}
      \item |\write18|、|\directlua| 与 Python\TeX{}
    \end{itemize}
\end{itemize}
\end{frame}

\begin{frame}[fragile]
\frametitle{深入字体}
\begin{itemize}
  \item<+-> NFSS 与字体的坐标

    \begin{itemize}
      \item 字体族、形状、系列、编码、字号
      \item \TeX{} 仅需要 metric 信息:|.tfm|
    \end{itemize}

  \item<+-> 现代方案:OpenType

    \begin{itemize}
      \item 编码层面:支持 Unicode
      \item 东亚文字:超大字符集、地区变体、竖排
      \item 中东、南亚文字:Bi-di 文本、上下文连字、字符序调整
    \end{itemize}

  \item<+-> OpenType 中的数学支持(|MATH| 表)

    \begin{itemize}
      \item Unicode Math:字母、符号支持
      \item 「数学常数」:整体度量信息——上下标位置、分数线粗细等
      \item |MathVariants|:大小替换(积分号、根号、括号等)
      \item |MathGlyphConstruction|:字符形装配(更大的根号、括号等)
    \end{itemize}

  \item<+-> 可变字体\zhparen{\includegraphics{example/variable-font.pdf}}、\jatext{絵文字}%
    \zhparen{emoji, \raisebox{-.2ex}{\includegraphics{example/emoji.pdf}}}……
\end{itemize}
\end{frame}

\begin{frame}[fragile]
\frametitle{编写宏包}
\begin{itemize}
  \item<+-> 文学编程

    \begin{itemize}
      \item 代码、注释与文档合为一体(|.dtx| 文件)
      \item 使用 \pkg{doc} 与 \pkg{docstrip} 宏包
    \end{itemize}

  \item<+-> 发布

    \begin{itemize}
      \item 上传 CTAN 实际上并无门槛
      \item 但仍有必要了解:

        \begin{itemize}
          \item \TeX{} 目录结构(TDS)
          \item 测试系统:\pkg{l3build} 宏包
          \item 版本控制、持续集成
          \item 许可证选择:\LaTeX{} 内核使用 LPPL 1.3c
            \link{https://www.latex-project.org/lppl/lppl-1-3c}
        \end{itemize}
    \end{itemize}

  \item<+-> 参考

    \begin{itemize}
      \item \LaTeX{} 内核代码:\pkg{source2e.pdf}、\pkg{classes.pdf}、\pkg{source3.pdf}
      \item 书籍:\emph{The \TeX book}、\emph{\TeX{} by Topic}、\emph{The \LaTeX{} Companion} 等
    \end{itemize}
\end{itemize}
\end{frame}

\begin{frame}[fragile]
\frametitle{宏包推荐}
\footnotesize
\setbeamertemplate{itemize/enumerate subbody begin}{\scriptsize}
\setlength{\leftmarginii}{1.5em}
\begin{multicols}{3}
  \begin{itemize}
    \item 必备

      \begin{itemize}
        \item \pkg{amsmath}
        \item \pkg{graphicx}
        \item \pkg{hyperref}
      \end{itemize}

    \item 样式

      \begin{itemize}
        \item \pkg{caption}
        \item \pkg{enumitem}
        \item \pkg{fancyhdr}
        \item \pkg{footmisc}
        \item \pkg{geometry}
        \item \pkg{ntheorem}
        \item \pkg{titlesec}
      \end{itemize}

    \item 数学

      \begin{itemize}
        \item \pkg{bm}
        \item \pkg{mathtools}
        \item \pkg{physics}
        \item \pkg{unicode-math}
      \end{itemize}

    \item 表格

      \begin{itemize}
        \item \pkg{array}
        \item \pkg{booktabs}
        \item \pkg{longtable}
        \item \pkg{tabularx}
      \end{itemize}

    \item 插图、绘图

      \begin{itemize}
        \item \pkg{float}
        \item \pkg{pdfpages}
        \item \pkg{standalone}
        \item \pkg{subfig}
        \item \pkg{pgf}/\pkg{tikz}
        \item \pkg{pgfplots}
      \end{itemize}

    \item 字体

      \begin{itemize}
        \item \pkg{newtx}
        \item \pkg{newpx}
        \item \pkg{pifont}
        \item \pkg{fontspec}
      \end{itemize}

    \item 多语言

      \begin{itemize}
        \item \pkg{babel}
        \item \pkg{polyglossia}
        \item \pkg{ctex}
        \item \pkg{xeCJK}
        \item \pkg{luatexja}
      \end{itemize}

    \item 杂项功能

      \begin{itemize}
        \item \pkg{algorithm2e}
        \item \pkg{beamer}
        \item \pkg{biblatex}
        \item \pkg{fancyhdr}
        \item \pkg{listings}
        \item \pkg{mhchem}
        \item \pkg{microtype}
        \item \pkg{minted}
        \item \pkg{natbib}
        \item \pkg{siunitx}
        \item \pkg{xcolor}
      \end{itemize}
  \end{itemize}
\end{multicols}
\vspace*{-0.5cm}
\end{frame}

\begin{frame}[standout]
  \huge \textbf{请务必先读文档!} \\[1ex] \pause
  \footnotesize 命令行执行 \texttt{texdoc \textit{package}}
\end{frame}

\begin{frame}[fragile]
\frametitle{Markdown}
\begin{columns}
\lstset{%
  moredelim    = [s][emphstyle]{*}{*},
  moredelim    = [s][keywordstyle]{**}{**},
  moredelim    = [s][emphstyle2]{\`}{\`},
  moredelim    = [s][emphstyle2]{\`\`\`}{\`\`\`},
  moredelim    = [l][keywordstyle2]{\#},
  moredelim    = [is][keywordstyle2]{+}{+},
  moredelim    = *[is][\itshape]{!}{!},
  moredelim    = [is][keywordstyle]{(+}{+)},
  moredelim    = [is][emphstyle2]{(-}{-)},
  basicstyle   = \scriptsize\ttfamily,
  keywordstyle = [1]\bfseries\color{keyword},
  keywordstyle = [2]\bfseries\color{texcs},
  emphstyle    = [1]\itshape\color{emph1},
  emphstyle    = [2]\color{inline}}
\begin{column}{0.48\textwidth}
  \begin{lstlisting}[gobble=2]
  # Markdown syntax

  This is **bold text**.
  This text is *italicized*.
  Use `git status` to list all
  new or modified files.

  Block code:

  ```
  git status
  git add
  git commit
  ```

  Quotation:

  +>+ !Markdown uses email-style `>`!
  +>+ !characters for blockquoting.!
  \end{lstlisting}
\end{column}
\begin{column}{0.48\textwidth}
  \begin{lstlisting}[gobble=2]
  ## List

  ### Bullet list

  +*+ apples
  +*+ oranges
  +*+ pears

  ### Numbered list

  +1.+ wash
  +2.+ rinse
  +3.+ repeat

  +---+

  Link: from [(+Wikipedia+)]
  ((-https://en.wikipedia.org/wiki/-)
  (-Markdown-))

  \end{lstlisting}
\end{column}
\end{columns}
\vspace{-0.6cm}
\end{frame}

\begin{frame}[fragile]
\frametitle{Git}
\begin{itemize}
  \item<+-> 版本管理的必要性

    \begin{itemize}
      \item 远离「初稿,第二稿,第三稿……终稿,终稿(打死也不改了)」
      \item 有底气做大范围修改、重构
      \item 方便与他人协同合作
    \end{itemize}

  \item<+-> 基本用法

    \begin{itemize}
      \item 把大象放进冰箱:|git init|、|git add|、|git commit|
      \item 时空穿梭:|git reset|、|git revert|
      \item 平行宇宙:|git branch|、|git checkout|、|git rebase|
      \item 推荐用 VS Code 等进行可视化操作
      \item 参考链接:\link{https://git-scm.com/book/en/v2}
        \link{https://www.liaoxuefeng.com/wiki/0013739516305929606dd18361248578c67b8067c8c017b000}
    \end{itemize}

  \item<+-> GitHub \href{https://github.com}{\faGithub}

    \begin{itemize}
      \item 远程 Git 仓库
      \item Clone \& fork
      \item Issues \& pull requests
      \item<+-> \alert{提醒:绑定 \texttt{.edu} 邮箱可以免费无限量使用私有仓库}
    \end{itemize}
\end{itemize}
\end{frame}

\begin{frame}{获取帮助}
\begin{itemize}
  \item<+-> 搜索、提问的姿势

  \begin{itemize}
    \item 优先使用英文 + Google (if possible)
    \item 提供最小工作示例(MWE, minmal working example)
      \begin{itemize}
        \item 能复现问题
        \item 尽量不带冗余内容
      \end{itemize}
  \end{itemize}

  \item<+-> 在线社区

  \begin{itemize}
    \item \TeX{} - \LaTeX{} Stack Exchange \link{https://tex.stackexchange.com}
    \item \CTeX{} 临时论坛 \link{https://github.com/CTeX-org/forum}
    \item \LaTeX{} 工作室 \link{https://www.latexstudio.net}
      \begin{itemize}
        \item 资源需要鉴别,且部分内容需付费
      \end{itemize}
  \end{itemize}
\end{itemize}
\end{frame}

\begin{frame}[fragile]
\frametitle{参考文献与扩展阅读}
\begin{multicols}{2}
\tiny
\newcommand\BOOK[1]{\textbf{#1}}
\newcommand\TAG[1]{\CASE{[#1]}}
\begin{thebibliography}{99}
  \bibitem{}
    \textsc{Knuth D E}.
    \newblock \BOOK{The \TeX book: Computers \& Typesetting, volume C} \TAG{M}, 1984.
    \newblock Boston: Addison--Wesley Publishing Company
  \bibitem{}
    刘海洋.
    \newblock \BOOK{\LaTeX{} 入门} \TAG{M}, 2013.
    \newblock 北京:电子工业出版社
  \bibitem{}
    \jatext{高冈昌生}.\\
    翻译:刘庆,监修:陈嵘.
    \newblock \BOOK{西文排版:排版的基础和规范} \TAG{M}, 2016.
    \newblock 北京:中信出版集团
  \bibitem{}
    \textsc{Oetiker T}, \textsc{Partl H}, \textsc{Hyna I} and \textsc{Schlegl E}.\\
    翻译:\CTeX{} 开发小组.
    \newblock \BOOK{一份(不太)简短的 \LaTeXe{} 介绍:或 112 分钟了解 \LaTeXe{}} \TAG{EB/OL}, 2020.
    \newblock \url{https://ctan.org/pkg/lshort-zh-cn}
  \bibitem{}
    黄新刚(包太雷).
    \newblock \BOOK{\LaTeX{} Notes: 雷太赫排版系统简介(第二版)} \TAG{EB/OL}, 2019.
    \newblock \url{https://github.com/huangxg/lnotes}
  \bibitem{}
    汪彧之,陈晟祺.
    \newblock \BOOK{如何使用 \LaTeX{} 排版论文} \TAG{EB/OL}, 2021.
    \newblock \url{https://github.com/tuna/thulib-latex-talk}
  \bibitem{}
    刘海洋.
    \newblock \BOOK{\LaTeX{} \CJKsout[thickness=0.1em]{快速}入门} \TAG{EB/OL}, 2020.
    \newblock Video: \href{https://www.bilibili.com/video/BV1s7411U7Pr}{\faVideo}
    % Old version: https://bbs.pku.edu.cn/attach/e7/f2/e7f2bb698b9c3672/tex_intro_talk.pdf
  \bibitem{}
    林莲枝.
    \newblock \BOOK{漫谈 \LaTeX{} 排版常见概念误区:别把 \LaTeX{} 当 Word 用!}\TAG{EB/OL}, 2018.
    \newblock Video: \href{https://www.bilibili.com/video/BV1r4411o7KJ}{\faVideo}\quad
      PDF: \href{http://static.latexstudio.net/wp-content/uploads/2018/03/LianTze-presentation-0320-forReading.pdf}{\faDownload}
  \bibitem{}
    Wikibooks.
    \newblock \BOOK{\LaTeX{}---Wikibooks, The Free Textbook Project} \TAG{EB/OL}.
    \newblock \url{https://en.wikibooks.org/wiki/LaTeX}
  \bibitem{}
    Overleaf.
    \newblock \BOOK{Overleaf Documentation} \TAG{EB/OL}.
    \newblock \url{https://www.overleaf.com/learn}
  \bibitem{}
    \LaTeX{} project.
    \newblock \BOOK{Learn\LaTeX.org} \TAG{EB/OL}.
    \newblock \url{https://www.learnlatex.org}
\end{thebibliography}
\end{multicols}
\end{frame}
