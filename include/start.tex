\section{开始之前……}

\begin{frame}[fragile]
\frametitle{命令行基础}
\begin{itemize}
  \item 打开终端
    \begin{itemize}
      \item \faWindows{}:右键开始菜单、空白处 \kbd{Shift} + 右键、\kbd{Windows} + \kbd{R} \& |cmd|
      \item \faLinux{}:\kbd{Ctrl} + \kbd{Alt} + \kbd{T}
      \item \faApple{}:\kbd{⌘} + \kbd{Space} 搜索 Terminal、可在 Finder 中添加服务
    \end{itemize}
  \item 基本命令:
    \begin{itemize}
      \item |cd|、|ls/dir|、|rm/del|、|clear/cls|
      \item 选项:|-h|、|--help|、|/?|
    \end{itemize}
  \item 其他:
    \begin{itemize}
      \item 复制粘贴:\kbd{Ctrl}/\kbd{Shift} + \kbd{Ins}、\kbd{Ctrl}/\kbd{⌘} + \kbd{C}/\kbd{V}、
      \item 路径连接符:斜线(|/|)或反斜线(|\|)
      \item 换行符:LF(|\n|)或 CRLF(|\r\n|)
      \item 进程终结者:\kbd{Ctrl} + \kbd{C}
    \end{itemize} \pause
  \item \alert{尽量不要用中文}
\end{itemize}
\end{frame}

\begin{frame}{编码}
关于 \LaTeX{} 源文件的编码,我们给出如下结论:\pause
\begin{alertblock}{编码定理}
  一般地,在任何场合使用(不带 BOM)的 \alert{UTF\CASE{-}8} 编码均是最优选择.
\end{alertblock} \pause
此定理的证明留做习题.
\end{frame}
